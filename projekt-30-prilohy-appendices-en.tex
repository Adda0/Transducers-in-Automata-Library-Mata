% This file should be replaced with your file with an appendices (headings below are examples only)

% For compilation piecewise (see projekt.tex), it is necessary to uncomment it and change
%\documentclass[../projekt.tex]{subfiles}
%\begin{document}

% Placing of table of contents of the memory media here should be consulted with a supervisor
\chapter{Contents of the included storage media}

The following list lists the contents of the included storage media. Listed are
only the few top-level folders in the folder hierarchy.

\begin{itemize}
	\item \texttt{mata/}: The main folder containing the source code for \mata automata library, with the reference implementation of finite transducers.
    \item \texttt{docs/}: The LaTeX source files for this work.
    \item \texttt{benchmarks/}: The benchmarks generated from runs of \noodler on SMT-LIB benchmarks, used in comparison of \mata with \mona.
    \item \texttt{experiments/}: The experimental pipeline with the source code for running experiments comparing \mata with \mona, source code for \mona.
    \begin{itemize}
        \item \texttt{results/raw/}: The storage of raw generated CSV files when running experiments.

        \item \texttt{results/processed/}: The CSV files with appropriate names prepared for analysis for experiments shown in this work.

        \item \texttt{analysis}: The scripts for analysing run experiments (generate graphs and tables shown in this work).
        \begin{itemize}
          \item \texttt{plots/}: The graphs and plots generated from analysis of the run experiments shown in this work.
        \end{itemize}
    \end{itemize}
\end{itemize}

% \chapter{Manual}
\chapter{Reference Implementation Manual}

Here, we describe how our reference implementation for transducers in \mata can be run, tested, and how one can reproduce the experiments shown in this work.


% TODO.

The experiments shown in this work were run on GNU/Linux system Ubuntu 22.04.4 LTS (with the Linux kernel GNU/Linux 5.15.0-106-generic x86\_64).
The experiments should be runnable on any Unix-like system, provided the following requirements are met.

The following programs are required for the implementation and experimental pipeline to run:
\begin{itemize}
    \item Python 3.10.12 or higher,
    \item C++ compiler, experiments run on g++ (Ubuntu 11.4.0-1ubuntu1\textasciitilde 22.04) 11.4.0,
    \item cmake 3.22.1 or higher,
    \item
\end{itemize}

The experiments can be run as follows:
\begin{itemize}
  \item Compile and install \mata with finite transducers:
  \\
  \texttt{make -C mata release}
  \\
  \texttt{sudo make -C mata install}

  \item Compile programs in experimental pipeline:
  \\
  \texttt{cd experiments/}
  \\
  \texttt{make -C mona}
  \\
  \texttt{make -C mata}

  \item Run an experiment:
  \\
  \texttt{
    ./run\_on\_benchmark.py --runs <RUNS> --timeout <TIMEOUT> <OPERATION> <PATH>
  }
  where \texttt{<RUNS>} is to be replaced with the number of runs on each benchmark instance should be run; \texttt{<TIMEOUT>} is the requested timeout for each benchmark instance in seconds; \texttt{<OPERATION>} is the benchmark operation to run; and \texttt{<PATH>} is the path to a file containing a benchmark instance in \texttt{.mata} format, or a folder containing (possibly inside additional subfolders) files with the benchmark instances in \texttt{.mata} format to run the experiments on.

  \item Run all experiments from this work:
  \\
  \texttt{./run\_all\_experiments.sh}

  \item See generated CSV files with results in \texttt{results/raw/}. Copy the files over to \texttt{results/processed/} with descriptive names.

  \item Analyse the results:
  \\
  \texttt{cd experiments/analyse}
  \\
  \texttt{python -m venv .venv}
  \\
  \texttt{source .venv/bin/activate}
  \\
  \texttt{pip install -r requirements.txt}
  \\
  \texttt{analyse.py}
  where \texttt{analyse.py} expects the following CSV files with results in \texttt{results/processed/}:
  \begin{itemize}
    \item \texttt{
    results/processed/<BENCHMARK>\_apply\_language.csv
    }
    \item \texttt{
      results/processed/<BENCHMARK>\_apply\_literal.csv
    }

    \item \texttt{
results/processed/<BENCHMARK>\_projection.csv
    }

    \item \texttt{
results/processed/<BENCHAMRK>\_composition.csv
    }
  \end{itemize}
  where \texttt{<BENCHAMRK>} is replaced with \texttt{transducer-plus} and \texttt{webapp}.

\end{itemize}


%\chapter{Configuration file}

%\chapter{Scheme of RelaxNG configuration file}

%\chapter{Poster}

% \noindent \textbf{Bibliographic citation}

% \medskip

% \noindent \textsc{Nováková}, J. \textit{Web on writing theses and dissertations} online. Edited by Jan NOVÁK. version 1.0. Brno: Brno University of Technology, Faculty of information technology, 2. February 1998 14:10. revised 12. 2. 2020. ISSN 1234-5678. Available at: \url{https://doi.org/10.1000/BC1.0}. [cit. 2020-02-12]. This is a made up citation.

% For compilation piecewise (see projekt.tex), it is necessary to uncomment it
%\end{document}
